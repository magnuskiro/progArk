% latex article template

% cheat sheet(eng): http://www.pvv.ntnu.no/~walle/latex/dokumentasjon/latexsheet.pdf
% cheat sheet2(eng): http://www.pvv.ntnu.no/~walle/latex/dokumentasjon/LaTeX-cheat-sheet.pdf
% reference manual(eng): http://ctan.uib.no/info/latex2e-help-texinfo/latex2e.html

% The document class defines the type of document. Presentation, article, letter, etc. 
\documentclass[12pt, a4paper]{article}

% packages to be used. needed to use images and such things. 
\usepackage[pdfborder=0 0 0]{hyperref}
\usepackage[utf8]{inputenc}
\usepackage[english]{babel}
\usepackage{graphicx}
\PassOptionsToPackage{hyphens}{url}

% hides the section numbering. 
\setcounter{secnumdepth}{-1}

% Graphics/image lications and extensions. 
\DeclareGraphicsExtensions{.pdf, .png, .jpg, .jpeg}
\graphicspath{{./images/}}


\title{
	ATAM Document \\
    TDT4240 - Group A14 \\
	\\
	}
}
\author{
	\underline{Group members:} \\
    Bremnes, Jan A. S.\\
    Johanessen, Stig Tore\\
	Hesselberg, Håkon \\
    Kirø, Magnus L.\\
	Randby, Simon \\
    Tørresen, Håvard\\
}

\date{\today}

\begin{document}
\maketitle
\pagenumbering{arabic}

\tableofcontents

\section{Introduction}
The evaluated group of this report is: A13. 
Group A13 have decided to focus their project on modifiability and testability. 

\section{Attribute utility tree}
\begin{tabular}{ p{.2\textwidth} p{.3\textwidth} p{.1\textwidth}
p{.4\textwidth} }
Attribute & Scenario & Priority & Details \\
Testability & T1 & H & - \\
Modifiability & M1 & L & - \\
Modifiability & M2 & H & - \\
Modifiability & Change COTS & (M,L) & Semantic coherence and decoupling.
Decoupling is good and it’s easily combined with modularity, but keep in mind that it’s not the same thing. \\
Testability & The verification of working game features & (M,L) & Limited Structural complexity might not improve the testability of the program. \\
\end{tabular}

\section{Analysis of Architectural Approach}

\subsection{Scenario}
\begin{itemize}
	\item Scenario #: M1
	\item Scenario name: Add game mode
	\item Attribute(s): Modifiability
	\item Environment: Build time
	\item Stimulus: Add game mode
	\item Response: Modification is made
\end{itemize}

\subsubsection{Architectural Decisions}
\begin{tabular}{ p{.2\textwidth} p{.2\textwidth} p{.2\textwidth}
p{.2\textwidth} p{.2\textwidth} }
Architectural Decisions & Sensitivity & Tradeoff & Risk & NonRisk \\
 & & & & \\
Modifiability by being able to add diverse content & Could possibly interfere
with the environment & Tradeoff with single, unifying, marketable, game mode
& Reduce game functionality and richness & N1\\
Reduce coupling & S1, S3 & T1 & -  & N1 \\
Keep semantic coherence & S1, S2 & T2 & - &  N1 \\
Limit structural complexity & S1, S3 & T3 & - &  N1 \\
\end{tabular}

\subsubsection{Reasoning:}
Because of the importance of game modes this is choosen as a scenario. 

\subsection{Scenario}
\begin{itemize}
    \item Scenario #: M2
    \item Scenario name: Add landscape generator
    \item Attribute(s): Modifiability
    \item Environment: Build time
    \item Stimulus: Add landscape generator
    \item Response: Modification is made
\end{itemize}

\subsubsection{Architectural Decisions}
\begin{tabular}{ p{.2\textwidth} p{.2\textwidth} p{.2\textwidth}
p{.2\textwidth} p{.2\textwidth} }
Architectural Decisions & Sensitivity & Tradeoff & Risk & NonRisk \\
Expandability, Diversity, Choice & Changes to how the game reads landscapes &
Will be hard to change the way terrains are implemented later & Won’t require
rewriting when adding more terrains later. & N1 \\
\end{tabular}

\subsubsection{Reasoning:}
Users should be able to choose which generator to use, so it should be easy for
the developer to add new ones.

\subsection{Scenario}
\begin{itemize}
	\item Scenario #: T1
    \item Scenario name: Immediate effects of game settings change
    \item Attribute(s): Testability
    \item Environment: Run-time
    \item Stimulus: Changing game settings
    \item Response: Gameplay is adjusted to changes
\end{itemize}

\subsubsection{Architectural Decisions}
\begin{tabular}{ p{.2\textwidth} p{.2\textwidth} p{.2\textwidth}
p{.2\textwidth} p{.2\textwidth} }
Architectural Decisions & Sensitivity & Tradeoff & Risk & NonRisk \\
Use libGDX & S1, S2 & T1 & R1 & N1 \\
\end{tabular}

\subsubsection{Reasoning:} 
Tests are good because knowledge of the program is needed for a productive
development. The risk is that the test might crash things if not done properly,
and that is very bad if it happens at run-time. Tradeoff is that testing
consumes time that could have been spent on development. A NonRisk is that it
will most likely not hurt the tester, developers, or other stakeholders.

\section{Sensitivity Points} 
(how architectural decisions affects either modifiability or testability)
\begin{itemize}
    \item S1: affects testability positively. 
    \item S2: affects modifiability negatively.
    \item S3: affects modifiability positively. 
    \item S4: affect testability negatively. 
\end{itemize}

\section{Tradeoff Points}
\begin{itemize}
    \item T1: No apparent trade off. 
    \item T2: trades code readability for implementation time. 
    \item T3: trades functionality for modifiability. 
\end{itemize}

\section{Risks and non-risks}
\begin{itemize}
    \item R1: libGDX turns out to be hard to use properly
    \item N1: No apparent risks. 
\end{itemize}

\section{Own Experiences from using ATAM}
We were reminded about some functional requirements that we had not included in
our own design, mainly because of the fact that we did not create an exhaustive
list of requirements. 

\section{Problems and issues}
The low level of detail in the architecture and requirements documents, made it
difficult to write a comprehensive report. 

\section{Change log}
none

\end{document}
