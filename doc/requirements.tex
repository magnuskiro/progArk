% latex article template

% cheat sheet(eng): http://www.pvv.ntnu.no/~walle/latex/dokumentasjon/latexsheet.pdf
% cheat sheet2(eng): http://www.pvv.ntnu.no/~walle/latex/dokumentasjon/LaTeX-cheat-sheet.pdf
% reference manual(eng): http://ctan.uib.no/info/latex2e-help-texinfo/latex2e.html

% The document class defines the type of document. Presentation, article, letter, etc. 
\documentclass[12pt, a4paper]{article}

% packages to be used. needed to use images and such things. 
\usepackage[pdfborder=0 0 0]{hyperref}
\usepackage[utf8]{inputenc}
\usepackage[english]{babel}
\usepackage{graphicx}
\PassOptionsToPackage{hyphens}{url}

% hides the section numbering. 
\setcounter{secnumdepth}{-1}

% Graphics/image lications and extensions. 
\DeclareGraphicsExtensions{.pdf, .png, .jpg, .jpeg}
\graphicspath{{./images/}}

\title{
	Requirements Document \\
    TDT4240 - Group A14 \\
	\\
	\normalsize{
	\underline{Framework:} \\ 
	Android\\
    \\
	\underline{Quality Attributes:} \\
	Primary: Modifiability \\
	Secondary: Useability \\ 
    \\
	}
}
\author{
	\underline{Group members:} \\
    Bremnes, Jan A. S.\\
    Johanessen, Stig Tore\\
	Hesselberg, Håkon \\
    Kirø, Magnus L.\\
	Randby, Simon \\
    Tørresen, Håvard\\
}
\date{\today}

\begin{document}
\maketitle
\pagenumbering{arabic}

\newpage
\tableofcontents
\newpage

\section{introduction}
* description of the project and the phase (requirement)
* of game project: description of game concept should be sufficiently described and explained here. 
* structure of the document. 

\section{Functional requirements}
*A complete list of functionality requirements you have to fulfill in order to complete the task. Each requirement must have a unique ID. Can also be decomposed into sub-requirements. 

\section{quality requirements(scenarioes)}
* Write at leat scenarios for the most relevant quality attributes (modifiability, testability, safety, usability, etc) 
* Use (textual/table) scenarios of the type used in chapter 4 of the book.
* Make the quality requirements measurable/testable with some values that can e checked later. Make estimates for the response mearsure. 
* Tables are recommended to specify quality requirements.
* every quality requirement mus have an ID.

\section{COTS components and technical constraints}
describe the constraints  your architecture has due to your choice of COTS(android), If you have some other constraints relevant for your project, it should be stated here. 

\section{references}
list references. 

\section{issues}
optional point of issues you faced working with this of the project and the document. 

\section{changes}
to be described carried out with this document from first draft until final delivary including all improvements based on feedback from course staff and others. 

\end{document}
